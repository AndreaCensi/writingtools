
% Note: LyX uses the three styles "definition", "plain", and "remark". Remark and plain are in italic:
%
%   \theoremstyle{definition}
%   \newtheorem{defn}{\protect\definitionname}
%   \theoremstyle{plain}
%   \newtheorem{conjecture}{\protect\conjecturename}
%   \theoremstyle{plain}
%   \newtheorem{lem}{\protect\lemmaname}
% \theoremstyle{plain}
% \newtheorem{thm}{\protect\theoremname}
%   \theoremstyle{definition}
%   \newtheorem{problem}{\protect\problemname}
%  \theoremstyle{definition}
%   \newtheorem{example}{\protect\examplename}
%   \theoremstyle{plain}
%   \newtheorem{prop}{\protect\propositionname}
%   \theoremstyle{remark}
%   \newtheorem{rem}{\protect\remarkname}


\newtheoremstyle{plain}%                % Name
  {}%                                     % Space above
  {}%                                     % Space below
  {}%                             % Body font
  {}%                                     % Indent amount
  {\bfseries}%                            % Theorem head font
  {.}%                                    % Punctuation after theorem head
  { }%                                    % Space after theorem head, ' ', or \newline
  {}%                                     % Theorem head spec (can be left empty, meaning `normal')


\newtheoremstyle{remark}%                % Name
  {}%                                     % Space above
  {}%                                     % Space below
  {}%                             % Body font
  {}%                                     % Indent amount
  {\bfseries}%                            % Theorem head font
  {.}%                                    % Punctuation after theorem head
  { }%                                    % Space after theorem head, ' ', or \newline
  {}%                                     % Theorem head spec (can be left empty, meaning `normal')
